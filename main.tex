\documentclass{article}
\usepackage{graphicx} % Required for inserting images
\usepackage{geometry}
\usepackage{amsmath}
\usepackage{amsfonts}

\geometry{
    a4paper,
    left=20mm,
    right=20mm,
    top=20mm,
    bottom=20mm
}

\title{fo hw01}
\author{cs}
\date{October 2025}

\begin{document}

\maketitle

%%%%%%%%%%%%%%%%%%%%%%%%%%%%%%%%%%%%%%%%%%%%%%%%%%%%%%%%%%%%%%%%%%%%%%%%%%%%%%%%
\section{Question 5}

%%%%%%%%%%%%%%%%%%%%%%%%%%%%%%%%%%%%%%%%%%%%%%%%%%%%%%%%%%%%%%%%%%%%%%%%%%%%%%%%
\subsection{5.1}
We first show that the epigraph of the following function is a polyhedron:
$$f(x)=\max_{i=1,\dots,m}\{(a^i)^T x+b_i\}.$$
Using the definition of the epigraph set:
$$\text{epi}(f)
=\{(x,z)\in\text{dom}(f)\times
\mathbb{R}\hspace{0.03in}|\hspace{0.03in}z\geq f(x)\}$$
we see that the condition $z\geq f(x)$ unpacks into:
$$z\geq\max_i\{(a^i)^T x+b_i\}$$
which in turn defines a sequence of inequalities:
$$z\geq(a^1)^T x+b_1$$
$$z\geq(a^2)^T x+b_2$$
$$\dots$$
$$z\geq(a^m)^T x+b_m.$$
Specifying more than one condition requires the intersection of their respective
partitions; each inequality of linear form defines a halfspace, and so we have an
intersection of a finite number of halfspaces, exactly the definition of a
polyhedron.

%%%%%%%%%%%%%%%%%%%%%%%%%%%%%%%%%%%%%%%%%%%%%%%%%%%%%%%%%%%%%%%%%%%%%%%%%%%%%%%%
\subsection{5.2}
We can use the fact that the function $f$ is convex if and only if the epigraph
of $f$, $\text{epi}(f)$, is convex (this is proposition 3.1 in the notes). Yet
we have shown previously that $\text{epi}(f)$ is a polyhedron, and all 
polyhedrons are by default convex (remark 6.2). We proceed to verify this for our
$\text{epi}(f)$. We need that $\forall\lambda\in[0,1]$ and
$\forall (x,t),(y,s)\in\text{epi}(f)$, that
$\lambda(x,t)+(1-\lambda)(y,s)\in\text{epi}(f)$, or that:
$$\lambda t+(1-\lambda)s\geq f\Bigl(\lambda x+(1-\lambda)y\Bigr)$$
which after substitution, our goal unpacks into:
$$\lambda t+(1-\lambda)s\geq
\max_i\{\lambda(a^i)^T x+(1-\lambda)(a^i)^T y+b_i\}$$
$$\lambda t+(1-\lambda)s\geq
\max_i\{\lambda\Bigl((a^i)^T x+b_i\Bigr)+
(1-\lambda)\Bigl((a^i)^T y+b_i\Bigr)\}.$$
This form is useful, since we already know that:
$$t\geq\max_i\{(a^i)^T x+b_i\}$$
$$s\geq\max_i\{(a^i)^T y+b_i\}$$
and using the fact that $\max{a}+\max{b}\geq\max{(a+b)}$:
\begin{align*}
    \lambda t+(1-\lambda)s
    &\geq\lambda\max_i\{(a^i)^T x+b_i\}+(1-\lambda)\max_i\{(a^i)^T y+b_i\} \\
    &\geq\max_i\{\lambda\Bigl((a^i)^T x+b_i\Bigr)+
    (1-\lambda)\Bigl((a^i)^T y+b_i\Bigr)\}
\end{align*}
which is exactly our initial goal. Therefore $\text{epi}(f)$ is convex, and so
$f$ is a convex function.

\end{document}